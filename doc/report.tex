%!TEX TS-program = xelatex
%!TEX encoding = UTF-8 Unicode

\documentclass[a4paper]{article}

\usepackage{xltxtra}
\usepackage{amsfonts}
\usepackage{polyglossia}
\usepackage{fancyhdr}
\usepackage{geometry}
\usepackage{dsfont}
\usepackage{amsmath}
\usepackage{amsthm}
\usepackage{amssymb}
\usepackage{physics}
\usepackage{mathtools}
\usepackage{bm}

\geometry{a4paper,left=15mm,right=15mm,top=20mm,bottom=20mm}
\pagestyle{fancy}
\lhead{Parker C. Lusk}
\chead{Autopilot Estimators}
\rhead{\today}
\cfoot{\thepage}

\setlength{\headheight}{23pt}
\setlength{\parindent}{0.0in}
\setlength{\parskip}{0.0in}

\newtheorem*{prop}{Proposition}
\newtheorem*{defn}{Definition}
\newtheorem*{thm}{Theorem}
\newtheorem*{cor}{Corollary}
\newtheorem*{lem}{Lemma}
\newtheorem*{rem}{Remark}

\DeclarePairedDelimiterX{\inn}[2]{\langle}{\rangle}{#1, #2}

\begin{document}
\section*{Overview}
High-level autonomy requires confidence in low-level control and estimation.
Autopilot firmware can use an inertial measurement unit (IMU) with three orthogonal rate gyroscopes and three orthogonal accelerometers to perform estimation.
If sensors were perfect, the estimator would simply integrate the gyro measurements to obtain an estimate of the attitude.
However, gyros tend to drift over time due to bias.
The accelerometer can be used to correct the drift in the roll and pitch axes, however yaw bias is unobservable with an IMU alone.
This document explores various schemes and implementations of attitude estimation using an IMU, with occasional external attitude updates from vision, motion capture, or some other source.

\section*{IMU Model}
With the advent of micro electro-mechanical systems (MEMS), the scale of IMUs has decreased significantly over the years.
In particular, the spread of small MEMS IMUs is in part due to the proliferation of smart phones and wearable electronics.

To understand the IMU model, consider a rigid body $\{B\}$ with a position $\bm{r}$ expressed in an inertial frame $\{A\}$.
An IMU is `strapped down' to the origin of the rigid body and is axis aligned with it.
In other words, the IMU sensor frame $\{S\}$ is identified with the body frame $\{B\}$.
Unless otherwise stated, we will make use of the East-North-Up (ENU) inertial frame with its corresponding Front-Left-Up (FLU) body frame.
The rotation matrix $R^A_B$ is the global-to-local rotation of $\{B\}$ w.r.t $\{A\}$ and it takes data from $\{B\}$ into $\{A\}$.

With this problem geometry, we write the measurements received by the IMU below.

\subsection*{Accelerometer}
Accelerometers measure specific the instantaneous linear acceleration (really the specific force) minus the force of the (conservative) gravitational field $\bm{F}_g = -g\bm{e}_3$.
\begin{equation}
    \bm{a} = \left(R^A_B\right)^\top (\ddot{\bm r} - \bm{F}_g) + \bm{\nu} + \bm{b}_a,
\end{equation}
where $\bm{\nu}$ is zero-mean Gaussian noise with variance $\Sigma^2_a$ and $\bm{b}_a$ is a bias term.
Accelerometers typically have a sample rate of 500 Hz or 1 kHz and 

\subsection*{Rate Gyro}
The measured angular rates $\bm{\omega} = \begin{bmatrix}p&q&r\end{bmatrix}^\top$ are the body rates of the vehicle with respect to the inertial frame, expressed in the body frame and is written as
\begin{equation}
    \bm{\omega} = \bm{\omega}_\text{true} + \bm{\eta} + \bm{b},
\end{equation}
where $\bm{\eta}$ is zero-mean Gaussian noise with variance $\Sigma^2_\omega$ and $\bm{b}$ is constant or slowly time-varying bias (i.e., $\dot{\bm b}\approx 0)$.

Gyros typically have a sample rate of 500 Hz, 1 kHz, or 8 kHz and are great at capturing high-frequency dynamics and are incredibly useful, especially over small time windows.
However, due to the low-frequency bias $\bm{b}$, simply integrating gyros over long timescales is unwise (e.g., more than tens of seconds).

\subsubsection*{Allan Variance}
\subsubsection*{AR(n) Model}

\section*{Complementary Filter}
One of the most basic estimators is the \textit{complementary filter}, also known as the \textit{balance filter} (Shane Colton, 2007).
The motivating principle is that gyros are good at capturing high-frequency dynamics, but have a low-frequency drift.
On the otherhand, accelerometers tend to have high-frequency noise (e.g., from non-smooth movement).
Further, when the system that the IMU is attached to has actuators (like a multirotor), 

\section*{Mahony's Nonlinear Complementary Filter}
One of the more popular estimators is Mahony's nonlinear complementary filter on $\mathrm{SO}(3)$.



\end{document}
